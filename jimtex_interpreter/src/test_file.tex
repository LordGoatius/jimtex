\documentclass{article}
\usepackage[dvips]{graphicx}
\usepackage{a4wide}
\usepackage{amsmath}
\usepackage{euscript}
\usepackage{amssymb}
\usepackage{amsthm}
\usepackage{amsopn}
\usepackage{multicol}
\usepackage{ stmaryrd }

\theoremstyle{definition}
\newtheorem*{definition}{Definition}
\newtheorem{theorem}{Theorem}

\newcommand{\vv}{\ensuremath{\vec{v}}}
\newcommand{\vu}{\ensuremath{\vec{u}}}
\newcommand{\vw}{\ensuremath{\vec{w}}}
\newcommand{\vx}{\ensuremath{\vec{x}}}
\newcommand{\vy}{\ensuremath{\vec{y}}}
\newcommand{\vb}{\ensuremath{\vec{b}}}
\newcommand{\vo}{\ensuremath{\vec{0}}}
\newcommand{\va}{\ensuremath{\vec{a}}}
\newcommand{\ve}{\ensuremath{\vec{e}}}

\newcommand{\R}{\mathbb{R}}
\newcommand{\Z}{\mathbb{Z}}
\newcommand{\C}{\mathbb{C}}
\newcommand{\N}{\mathbb{N}}
\newcommand{\Q}{\mathbb{Q}}
\newcommand{\Zn}{\mathbb{Z}/n\mathbb{Z}}

% *********************************************************************
\begin{document}
\begin{center}
\Large{Homework Set \# 6}
\#
#

\normalsize{Math 457: Abstract Algebra}

\vspace{0.2cm}

\hfill {\bf Name:} Your name here

\vspace{0.25cm}
\hrule
\end{center}

\vspace{0.3cm}

%%%%%%%%%%%%%%%%%%%%%%%%%%%%%%%%%%%%%%%%%%%%%%%%%%%%%%%%%%%%%%%%%%%%%%%%%%%%


\noindent \textbf{Instructions:}  Discuss and prove/solve \emph{all} of the following problems with your group.  Turn in a \LaTeX-ed copy of your solution to \textbf{Problem 2(a)}, \textbf{Problem 2(b)}, \textbf{Problem 2(c)} or \textbf{Problem 3(b)}.   \textbf{Also, submit a recording of your conversation :)}  


\bigskip

\noindent Homework \#6 Groups: 
\begin{multicols}{2}
\begin{itemize}
    \item[(1)] Harrison, Chris, Taden
    \item[(2)] Dayton, Jimmy, Zeke
    \item[(3)] Colin, Ben, Zach
    \item[(4)] Tyler, Ash, Indi
\end{itemize}
\end{multicols}
\vspace{1cm}

\noindent Problem 1:   %DF 7.6.4
Prove that if $R$ and $S$ are nonzero rings then $R\times S$ is never a field (even if $R$ and $S$ are fields).  \emph{Hint: Don't over think it.  It should be a \underline{very} short proof.}

\vspace{1cm}

%%%%%%%%%%%%%%%%%%%%%%%%%%%%%%%%%%%%%%%%%%%%%%%%%%%%%%%%%%%%%%%%%%%%%%%
%%%%%%%%%%%%%%%%%%%%%%%%%%%%%%%%%%%%%%%%%%%%%%%%%%%%%
\noindent Problem 2:  %DF Fraction Field Proof
Let $R$ be a commutative ring, $D$ be a nonempty subset of nonzero, non-zero divisors in $R$ which is closed under multiplication, and let $\mathcal{Q}$ be the \emph{ring of fractions of $D$ with respect to $R$}, as defined in class. 
 \begin{itemize}
     \item[(a)] Prove the ring operations defined on $\mathcal{Q}$ are well-defined
     \item[(b)] Prove $\mathcal{Q}$ is a commutative ring with unity
     \item[(c)] Prove the image of $D$ under the inclusion map ($i:R\rightarrow \mathcal{Q}$ defined by $i(r)=rd/d=[(rd,d)]\quad \forall r\in R$) is a subset of the group of units in $\mathcal{Q}$, (i.e. prove elements of $D$ ``become'' units in $\mathcal{Q}$)
\end{itemize}

\vspace{1cm}


%%%%%%%%%%%%%%%%%%%%%%%%%%%%%%%%%%%%%%%%%%%%%%%%%%%%%%%%%%%%%%%%%%%%%%%%%%
%%%%%%%%%%%%%%%%%%%%%%%%%






%%%%%%%%%%%%%%%%%%%%%%%%%%%%%%%%%%%%%%%%%%%%%%%%%%%%%%%%%%%%%%%%%%%%%%%%%%%%%
\noindent Problem 3  %Ideal Products
Given Ideals $A$ and $B$ of a commutative ring $R$, and an ideal $C$ of a ring $S$
\begin{itemize}
    \item[(a)]  Prove that $AB$ is an ideal of $R$.  You need to prove it is a subring first.
    \item[(b)] Prove $A\times C$ is an ideal of $R\times S.$ You need to pithily prove it is a subring first.
\end{itemize}

\vspace{1cm}

%%%%%%%%%%%%%%%%%%%%%%%%%%%%%%%%%%%%%%%%%%%%%%%%%%%%%%%%%%%%%%%%%%%%%%%
%%%%%%%%%%%%%%%%%%%%%%%%%%%%%%%%%%%%%%%%%%%%%%%%%%%%%%%%%%%%%%%%%%%%%%%%%%%%%%



\noindent Problem Extra Credit: submit a math related joke.  Points awarded on humor and clever integration of mathematical concepts.

\vspace{1cm}

\end{document}
